\documentclass[11pt]{article}
\usepackage[a4paper,margin=1in]{geometry}
\usepackage{amsmath, amssymb}
\usepackage{physics}
\usepackage{graphicx}
\usepackage{hyperref}
\usepackage{enumitem}

\title{Waves and Optics Formula Cheat Sheet}
\author{University Level}
\date{\today}

\begin{document}
\maketitle
\tableofcontents
\newpage

\section{Basic Wave Equations}
\subsection{General Wave Equation}
The standard form of the wave equation is:
\begin{equation}
    \nabla^2 \psi(\mathbf{r}, t) - \frac{1}{c^2} \frac{\partial^2 \psi(\mathbf{r}, t)}{\partial t^2} = 0,
\end{equation}
where \(\psi(\mathbf{r}, t)\) is the wave function and \(c\) is the wave speed.

\subsection{Plane Wave Solutions}
A common solution is the plane wave:
\begin{equation}
    \psi(x,t) = A e^{i(kx - \omega t)},
\end{equation}
with amplitude \(A\), wavenumber \(k\), and angular frequency \(\omega\) (where \( \omega = c k \) for non-dispersive media).

\section{Wave Parameter Conversions}
The following relationships are essential for converting between various wave parameters:
\begin{itemize}[leftmargin=*, label={--}]
    \item \textbf{Angular Frequency and Frequency:}
    \begin{equation}
        \omega = 2\pi f,
    \end{equation}
    where \(f\) is the frequency in hertz (Hz).
    
    \item \textbf{Frequency and Period:}
    \begin{equation}
        f = \frac{1}{T},
    \end{equation}
    where \(T\) is the period.
    
    \item \textbf{Wavenumber and Wavelength:}
    \begin{equation}
        k = \frac{2\pi}{\lambda},
    \end{equation}
    where \(\lambda\) is the wavelength.
    
    \item \textbf{Wave Speed Relation:}
    \begin{equation}
        c = \lambda f = \frac{\omega}{k}.
    \end{equation}
\end{itemize}

\section{Fourier Transforms}
The Fourier transform \( \mathcal{F}\{f(t)\}(\omega) \) and its inverse are defined as:
\begin{align}
    \mathcal{F}\{f(t)\}(\omega) &= \int_{-\infty}^{\infty} f(t)e^{-i\omega t}dt, \\
    f(t) &= \frac{1}{2\pi}\int_{-\infty}^{\infty} \mathcal{F}\{f(t)\}(\omega)e^{i\omega t}d\omega.
\end{align}

\subsection{Common Functions}
\begin{itemize}[leftmargin=*, label={--}]
    \item \textbf{Dirac Delta Function}:
    \begin{align}
        \mathcal{F}\{\delta(t)\}(\omega) &= 1.
    \end{align}
    
    \item \textbf{Rectangular Function} (often written as \(\operatorname{rect}\)):
    \begin{align}
        \operatorname{rect}\left(\frac{t}{T}\right) =
        \begin{cases}
            1, & |t| < \frac{T}{2}, \\
            \frac{1}{2}, & |t| = \frac{T}{2}, \\
            0, & |t| > \frac{T}{2}.
        \end{cases}
    \end{align}
    Its Fourier transform is:
    \begin{align}
        \mathcal{F}\left\{\operatorname{rect}\left(\frac{t}{T}\right)\right\}(\omega)
        &= T\, \operatorname{sinc}\left(\frac{\omega T}{2\pi}\right),
    \end{align}
    where \(\operatorname{sinc}(x) = \frac{\sin(\pi x)}{\pi x}\).

    \item \textbf{Gaussian Function}:
    \begin{align}
        f(t) &= e^{-at^2}, \quad a>0,
    \end{align}
    has the Fourier transform:
    \begin{align}
        \mathcal{F}\{e^{-at^2}\}(\omega)
        &= \sqrt{\frac{\pi}{a}}\, e^{-\frac{\omega^2}{4a}}.
    \end{align}
\end{itemize}

\section{Complex Numbers and the Complex Wave Function}
\subsection{Fundamental Properties}
\begin{itemize}[leftmargin=*, label={--}]
    \item \textbf{Euler's Formula:}
    \begin{equation}
        e^{i\theta} = \cos\theta + i\sin\theta.
    \end{equation}
    \item \textbf{Complex Conjugate:} For \(z = a + ib\), the conjugate is \(z^* = a - ib\).
    \item \textbf{Modulus:} The magnitude of \(z\) is given by
    \begin{equation}
        |z| = \sqrt{a^2+b^2}.
    \end{equation}
\end{itemize}

\subsection{Complex Wave Function}
In quantum mechanics and optics, waves are often represented as complex functions. A typical form is:
\begin{equation}
    \psi(x,t) = A e^{i(kx - \omega t)},
\end{equation}
where:
\begin{itemize}[leftmargin=*, label={--}]
    \item \(A\) is the amplitude,
    \item \(k\) is the wave number,
    \item \(\omega\) is the angular frequency.
\end{itemize}
The measurable \emph{intensity} or \emph{probability density} is:
\begin{equation}
    |\psi(x,t)|^2 = \psi(x,t)\psi^*(x,t),
\end{equation}
and for normalized functions, the condition is:
\begin{equation}
    \int_{-\infty}^{\infty} |\psi(x,t)|^2\, dx = 1.
\end{equation}

\section{Optics Essentials}
\subsection{Lens Equation}
The thin lens equation is given by:
\begin{equation}
    \frac{1}{f} = \frac{1}{d_o} + \frac{1}{d_i},
\end{equation}
where \(f\) is the focal length, \(d_o\) is the object distance, and \(d_i\) is the image distance.

\subsection{Huygens-Fresnel Principle}
Every point on a wavefront acts as a source of secondary spherical wavelets. The superposition of these wavelets forms the next wavefront. This principle is central to understanding diffraction and interference.

\subsection{Diffraction and Interference}
For example, the \textbf{Fraunhofer diffraction} pattern for a single slit of width \(a\) is:
\begin{equation}
    I(\theta) \propto \left( \frac{\sin(\beta)}{\beta} \right)^2, \quad \beta = \frac{\pi a \sin\theta}{\lambda},
\end{equation}
where \(\lambda\) is the wavelength of light.

\section{Additional Notes}
\begin{itemize}[leftmargin=*, label={--}]
    \item \textbf{Superposition Principle:} Linear wave equations allow the addition of solutions, which underlies interference patterns.
    \item \textbf{Dispersion Relation:} Relates \(\omega\) and \(k\) in media where wave speed depends on frequency.
    \item \textbf{Phase and Group Velocity:}
    \begin{align}
        v_p &= \frac{\omega}{k}, \\
        v_g &= \frac{d\omega}{dk}.
    \end{align}
\end{itemize}

\section{References and Further Reading}
For a deeper dive into these topics, consider reviewing textbooks on waves and optics, such as:
\begin{itemize}[leftmargin=*, label={--}]
    \item "Introduction to Electrodynamics" by David J. Griffiths,
    \item "Principles of Optics" by Born and Wolf,
    \item Standard texts on Fourier analysis.
\end{itemize}

\end{document}
