\documentclass[11pt]{article}
\usepackage[a4paper,margin=1in]{geometry}
\usepackage{amsmath, amssymb}
\usepackage{physics}
\usepackage{graphicx}
\usepackage{enumitem}

\begin{document}

\section{Fraunhofer Diffraction Example}

\subsection{Rectangular Aperture Transmission Function}
\begin{equation}
    T(x,y) = \operatorname{rect} \left(\frac{x}{2w}\right) \operatorname{rect} \left(\frac{y}{2h}\right).
\end{equation}

\subsection{Fraunhofer Approximation}
The amplitude immediately after the aperture at \( (x',y') \) is given by \( A \). At a distance \( z \), the far-field diffraction pattern is:

\begin{equation}
    U(x,y,z) = \frac{k}{iz} e^{ik_0 z} e^{\frac{ik_0}{2z} (x^2 + y^2)}
    \cdot \frac{1}{2\pi} \iint_{-\infty}^{\infty} U(x', y', z=0) e^{-i \frac{k_0}{z} x x'} e^{-i \frac{k_0}{z} y y'} dx' dy'.
\end{equation}

Since the aperture is limited to \( x' \in [-w, w] \) and \( y' \in [-h, h] \), we can express the integral as:

\begin{equation}
    U(x,y,z) = \frac{k}{iz} e^{ik_0 z} e^{\frac{ik_0}{2z} (x^2 + y^2)}
    \cdot \frac{1}{2\pi} A \int_{x'=-w}^{w} e^{-i \frac{k_0}{z} x x'} dx'
    \int_{y'=-h}^{h} e^{-i \frac{k_0}{z} y y'} dy'.
\end{equation}

\subsection{Using the Identity}
Using the identity:
\begin{equation}
    \int e^{-i\alpha x} dx = \frac{1}{-i\alpha} \left( e^{-i\alpha x} - e^{i\alpha x} \right) = \frac{2\sin(\alpha x)}{\alpha},
\end{equation}
we evaluate the integrals:

\begin{equation}
    U(x,y,z) = \frac{k_0}{iz} e^{ik_0 z} e^{\frac{ik_0}{2z} (x^2+y^2)}
    \cdot \frac{1}{2\pi} A \left( \frac{2\sin\left(\frac{k_0 w x}{z}\right)}{\frac{k_0 x}{z}} \right)
    \left( \frac{2\sin\left(\frac{k_0 h y}{z}\right)}{\frac{k_0 y}{z}} \right).
\end{equation}



\section{Double-Slit Interference}

\subsection{Aperture Function}
For two slits of width \( 2w \), separated by a distance \( d \), the transmission function is modeled as:
\begin{equation}
    T(x) = \operatorname{rect} \left(\frac{x - d/2}{2w} \right) + \operatorname{rect} \left(\frac{x + d/2}{2w} \right).
\end{equation}

\subsection{Using Fourier Transforms}
The intensity of the diffraction pattern is proportional to the squared magnitude of the Fourier transform:
\begin{equation}
    I \propto \left| \mathcal{F} \left[ \operatorname{rect} \left(\frac{x - d/2}{2w} \right) + \operatorname{rect} \left(\frac{x + d/2}{2w} \right) \right] \right|^2.
\end{equation}

\subsection{Evaluating the Fourier Transform}
Using the Fourier transform properties of the rectangular function:
\begin{equation}
    \mathcal{F} \left[ \operatorname{rect} \left( \frac{x}{2w} \right) \right] = \operatorname{sinc} \left(\frac{kx}{z} w\right),
\end{equation}
we find that the two slits contribute an additional phase factor due to their displacement:
\begin{equation}
    I \propto \left| \left( e^{i k x d / 2z} + e^{-i k x d / 2z} \right) \operatorname{sinc} \left( \frac{kx}{z} w \right) \right|^2.
\end{equation}

Since \( e^{i\theta} + e^{-i\theta} = 2\cos\theta \), we obtain:
\begin{equation}
    I = 4 \cos^2 \left( \frac{kx d}{2z} \right) \operatorname{sinc}^2 \left( \frac{kx}{z} w \right).
\end{equation}

\subsection{Interference Pattern}
The resulting pattern consists of:
\begin{itemize}
    \item A central \(\operatorname{sinc}^2(x)\) diffraction envelope due to the finite width of each slit.
    \item Rapid \(\cos^2(x)\) oscillations corresponding to interference from the two slits.
\end{itemize}
The resulting intensity pattern follows the behavior shown in the figure below.

\begin{figure}[h]
    \centering
    \includegraphics[width=0.6\textwidth]{double_slit_diagram.png}
    \caption{Double-slit diffraction pattern: The envelope follows a \(\operatorname{sinc}^2(x)\) function due to single-slit diffraction, while the fringes oscillate with \(\cos^2(x)\) due to interference.}
\end{figure}

\end{document}
