\documentclass[11pt]{article}
\usepackage[a4paper,margin=1in]{geometry}
\usepackage{amsmath, amssymb}
\usepackage{physics}
\usepackage{enumitem}
\usepackage{hyperref}
\usepackage{graphicx}

\title{Fourier Transform Cheat Sheet}
\author{University Level}
\date{\today}

\begin{document}
\maketitle
\tableofcontents
\newpage

\section{Fourier Transform: Definitions and Basic Properties}

The Fourier transform of a function \( f(t) \) is defined as:
\begin{equation}
    \mathcal{F}\{f(t)\}(\omega) = F(\omega) = \int_{-\infty}^{\infty} f(t)e^{-i\omega t}\, dt.
\end{equation}
The inverse Fourier transform is given by:
\begin{equation}
    f(t) = \frac{1}{2\pi}\int_{-\infty}^{\infty} F(\omega)e^{i\omega t}\, d\omega.
\end{equation}

Several useful properties include linearity, time-shifting, frequency-shifting, and the convolution theorem. In this sheet, we derive the Fourier transforms for some common functions.

\newpage

\section{Fourier Transform of the Dirac Delta Function}

The Dirac delta function, \(\delta(t)\), is defined such that:
\[
\int_{-\infty}^{\infty}\delta(t) \, dt = 1,
\]
and for any well-behaved function \(g(t)\),
\[
\int_{-\infty}^{\infty} g(t)\delta(t-t_0)\, dt = g(t_0).
\]

\subsection*{Derivation}
Starting from the definition:
\begin{align}
    \mathcal{F}\{\delta(t)\}(\omega) &= \int_{-\infty}^{\infty} \delta(t) e^{-i\omega t}\, dt.
\end{align}
Using the sifting property of the delta function (with \(t_0 = 0\)):
\begin{align}
    \mathcal{F}\{\delta(t)\}(\omega) &= e^{-i\omega\cdot0} = 1.
\end{align}

Thus, we have:
\begin{equation}
    \boxed{\mathcal{F}\{\delta(t)\}(\omega) = 1.}
\end{equation}

\newpage

\section{Fourier Transform of the Rectangular Function}

Define the rectangular function (also denoted as \(\operatorname{rect}(t/T)\)) by:
\begin{equation}
    \operatorname{rect}\left(\frac{t}{T}\right) =
    \begin{cases}
        1, & |t| < \frac{T}{2}, \\
        \frac{1}{2}, & |t| = \frac{T}{2}, \\
        0, & |t| > \frac{T}{2}.
    \end{cases}
\end{equation}

\subsection*{Derivation}
We compute the Fourier transform:
\begin{align}
    \mathcal{F}\left\{\operatorname{rect}\left(\frac{t}{T}\right)\right\}(\omega)
    &= \int_{-\infty}^{\infty} \operatorname{rect}\left(\frac{t}{T}\right) e^{-i\omega t}\, dt \\
    &= \int_{-T/2}^{T/2} e^{-i\omega t}\, dt.
\end{align}
Evaluating the integral:
\begin{align}
    \int_{-T/2}^{T/2} e^{-i\omega t}\, dt 
    &= \left[ \frac{e^{-i\omega t}}{-i\omega} \right]_{t=-T/2}^{T/2} \\
    &= \frac{1}{-i\omega} \left( e^{-i\omega (T/2)} - e^{i\omega (T/2)} \right).
\end{align}
Recall that:
\[
e^{-i\theta} - e^{i\theta} = -2i\sin\theta.
\]
Thus, we have:
\begin{align}
    \int_{-T/2}^{T/2} e^{-i\omega t}\, dt 
    &= \frac{1}{-i\omega} \left( -2i\sin\left(\frac{\omega T}{2}\right) \right) \\
    &= \frac{2\sin\left(\frac{\omega T}{2}\right)}{\omega}.
\end{align}

Often, this is written using the sinc function defined as:
\[
\operatorname{sinc}(x) = \frac{\sin x}{x}.
\]
Therefore,
\begin{equation}
    \boxed{\mathcal{F}\left\{\operatorname{rect}\left(\frac{t}{T}\right)\right\}(\omega) = T\, \operatorname{sinc}\left(\frac{\omega T}{2}\right),}
\end{equation}
where we have used the normalization:
\[
T\, \operatorname{sinc}\left(\frac{\omega T}{2}\right) = \frac{2\sin\left(\frac{\omega T}{2}\right)}{\omega}.
\]

\newpage

\section{Fourier Transform of the Gaussian Function}

Consider the Gaussian function:
\begin{equation}
    f(t) = e^{-at^2}, \quad a>0.
\end{equation}

\subsection*{Derivation}
The Fourier transform is given by:
\begin{align}
    \mathcal{F}\{e^{-at^2}\}(\omega) &= \int_{-\infty}^{\infty} e^{-at^2}e^{-i\omega t}\, dt \\
    &= \int_{-\infty}^{\infty} e^{-at^2 - i\omega t}\, dt.
\end{align}

To simplify the exponent, complete the square:
\[
-at^2 - i\omega t = -a\left(t^2 + \frac{i\omega}{a}t\right)
= -a\left[\left(t + \frac{i\omega}{2a}\right)^2 - \left(\frac{i\omega}{2a}\right)^2\right].
\]
Thus,
\begin{align}
    -at^2 - i\omega t &= -a\left(t + \frac{i\omega}{2a}\right)^2 + \frac{\omega^2}{4a}.
\end{align}
Then, the integral becomes:
\begin{align}
    \mathcal{F}\{e^{-at^2}\}(\omega)
    &= e^{-\frac{\omega^2}{4a}} \int_{-\infty}^{\infty} e^{-a\left(t+\frac{i\omega}{2a}\right)^2}\, dt.
\end{align}
Perform the change of variable:
\[
u = t + \frac{i\omega}{2a}, \quad du = dt.
\]
Since the shift is along the complex plane and the Gaussian decays rapidly, the integration limits remain \(-\infty\) to \(\infty\). Hence,
\begin{align}
    \int_{-\infty}^{\infty} e^{-a u^2}\, du = \sqrt{\frac{\pi}{a}}.
\end{align}
Thus, we obtain:
\begin{equation}
    \boxed{\mathcal{F}\{e^{-at^2}\}(\omega) = \sqrt{\frac{\pi}{a}}\, e^{-\frac{\omega^2}{4a}}.}
\end{equation}

\newpage

\section{Summary of Derived Fourier Transforms}

\begin{itemize}[leftmargin=*, label={--}]
    \item \textbf{Dirac Delta:}
    \[
    \mathcal{F}\{\delta(t)\}(\omega) = 1.
    \]
    
    \item \textbf{Rectangular Function:}
    \[
    \mathcal{F}\left\{\operatorname{rect}\left(\frac{t}{T}\right)\right\}(\omega) = T\, \operatorname{sinc}\left(\frac{\omega T}{2}\right) = \frac{2\sin\left(\frac{\omega T}{2}\right)}{\omega}.
    \]
    
    \item \textbf{Gaussian Function:}
    \[
    \mathcal{F}\{e^{-at^2}\}(\omega) = \sqrt{\frac{\pi}{a}}\, e^{-\frac{\omega^2}{4a}}.
    \]
\end{itemize}

\section{Table of Common Fourier Transforms}

The Fourier transform is defined as:
\begin{equation}
    \mathcal{F}\{f(t)\}(\omega) = F(\omega) = \int_{-\infty}^{\infty} f(t)e^{-i\omega t}\, dt.
\end{equation}

The inverse Fourier transform is given by:
\begin{equation}
    f(t) = \frac{1}{2\pi} \int_{-\infty}^{\infty} F(\omega)e^{i\omega t}\, d\omega.
\end{equation}

\bigskip
\begin{center}
    \renewcommand{\arraystretch}{1.8}
    \begin{tabular}{ |c|c| }
        \hline
        Function \( f(t) \) & Fourier Transform \( F(\omega) \) \\ 
        \hline\hline
        \( \delta(t) \) (Dirac Delta) & \( 1 \) \\ 
        \hline
        \( 1 \) & \( 2\pi \delta(\omega) \) \\
        \hline
        \( e^{i \omega_0 t} \) & \( 2\pi \delta(\omega - \omega_0) \) \\
        \hline
        \( e^{-a |t|}, \quad a > 0 \) & \( \frac{2a}{a^2 + \omega^2} \) (Lorentzian) \\
        \hline
        \( \operatorname{rect} \left( \frac{t}{T} \right) \) & \( T \operatorname{sinc} \left( \frac{\omega T}{2\pi} \right) \) \\
        \hline
        \( e^{-at^2} \) (Gaussian) & \( \sqrt{\frac{\pi}{a}} e^{-\frac{\omega^2}{4a}} \) \\
        \hline
        \( \cos(\omega_0 t) \) & \( \pi \left[ \delta(\omega - \omega_0) + \delta(\omega + \omega_0) \right] \) \\
        \hline
        \( \sin(\omega_0 t) \) & \( i\pi \left[ \delta(\omega - \omega_0) - \delta(\omega + \omega_0) \right] \) \\
        \hline
        \( e^{-a t} u(t), \quad a > 0 \) & \( \frac{1}{a + i\omega} \) \\
        \hline
        \( \frac{1}{t} \) (Hilbert Transform) & \( -i\pi \operatorname{sgn}(\omega) \) \\
        \hline
    \end{tabular}
\end{center}

\bigskip
\noindent Notes:
\begin{itemize}
    \item \( \operatorname{sinc}(x) = \frac{\sin(\pi x)}{\pi x} \).
    \item \( u(t) \) is the unit step function (Heaviside function).
    \item The **Dirac delta function** \( \delta(t) \) has the property \( \int_{-\infty}^{\infty} \delta(t) dt = 1 \).
\end{itemize}

\end{document}
